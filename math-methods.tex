%(c) 2013 Jyotirmoy Bhattacharya

%This document is licensed under the
%Creative Commons  Attribution-ShareAlike 3.0 Unported License
%See LICENSE file

\documentclass[11pt,reqno,openany]{amsbook}
\numberwithin{figure}{chapter}
\numberwithin{equation}{chapter}
\title{Mathematical Methods for Economics}
\author{Jyotirmoy Bhattacharya}
\email{jyotirmoy@jyotirmoy.net}
\thanks{\copyright\ 2013, Jyotirmoy Bhattacharya\\This work is licensed under a \href{http://creativecommons.org/licenses/by-sa/3.0/deed.en_US}{Creative Commons Attribution-ShareAlike 3.0 Unported License}.}
\date{\today}

\usepackage[paperwidth=6in,paperheight=9in,margin=0.7in]{geometry}
\usepackage{mathpazo}
\usepackage{hyperref}
\hypersetup{colorlinks=true,linkcolor=[rgb]{0.1,0.1,0.8}}
\usepackage{tikz}
\usetikzlibrary{positioning}

\theoremstyle{plain}
\newtheorem{thm}{Theorem}[chapter]
\newtheorem{fact}{Fact}[chapter]
\theoremstyle{definition}
\newtheorem{defn}{Definition}[chapter]
\newtheorem{xca}{Exercise}[chapter]
\newtheorem{xmpl}{Example}[chapter]

\renewcommand{\vec}[1]{\mathbf{#1}}
\newcommand{\mat}[1]{\mathbf{#1}}
\newcommand{\dotprod}[2]{{#1}\cdot{#2}}
\renewcommand{\Re}{\mathbb{R}}

\begin{document}
\frontmatter
\maketitle
\tableofcontents
\mainmatter
\chapter{Linear Algebra}
\section{Vectors}
We assume that you know how to calculate sums, scalar multiples and
inner products of vectors. We remind you below of the geometrical
interpretations of these operations and some properties that they
satisfy.

\begin{fact}[The Triangle Law for Vector Addition]
  Suppose
  \[\vec{c}=\vec{a}+\vec{b}\]
  If the vectors $\vec{a}$ and $\vec{b}$ are drawn as arrows with the
  tail of $\vec{b}$ touching the tip of $\vec{a}$ then $\vec{c}$ is an
  arrow with its tail touching the tail of $\vec{a}$ and its tip
  touching the tip of $\vec{b}$.
  
  \begin{figure}[h]
  \begin{tikzpicture}[scale=2]
    \draw [->] (0,0) -- (0,1.8);
    \draw [->] (0,0) -- (1.8,0);
    \draw [->, thick] (0,0) -- node[right=0.2] {$\vec{a}$} (1,0.5);
    \draw [->, thick] (1,0.5) -- node[right=0.2] {$\vec{b}$}  (1.5,1.5);
    \draw [->, thick] (0,0) -- node[left=0.2] {$\vec{c}$} (1.5,1.5);
  \end{tikzpicture}
  \caption{The Triangle Law}
  \end{figure}

  Vector addition has the following properties
  \begin{enumerate}
  \item $\vec{x}+(\vec{y}+\vec{z})=(\vec{x}+\vec{y})+\vec{z}$.
  \item $\vec{x}+\vec{y}=\vec{y}+\vec{x}$.
  \item There is a vector $\vec{0}$ such that for any vector
    $\vec{x}$,
    \[\vec{x}+\vec{0}=\vec{x}.\]
  \item For every vector $\vec{x}$ there is a vector $\vec{y}$ such
    that
    \[\vec{x}+\vec{y}=\vec{0}.\]
  \end{enumerate}
\end{fact}
%\vfill\pagebreak
\begin{fact}[Scalar Multiplication]
  For a vector $\vec{x}$ and a scalar $\alpha\ge 0$, the vector
  $\alpha\vec{x}$ lies in the same direction as $\vec{x}$ and has
  $\alpha$ times the length of $\vec{x}$ and the vector
  $-\alpha\vec{x}$ lies in the direction exactly opposite that of
  $\vec{x}$ and has $\alpha$ times the length.
  
  \begin{figure}[h]
  \begin{tikzpicture}[scale=2]
    \draw [<->] (-1,0) -- (1,0);
    \draw [<->] (0,-1) -- (0,1);
    \draw [->, thick] (0,0) -- 
        node[near end,right=0.2] {$\vec{x}$} (0.2,0.3);
    \draw [->, thick] (0,0) -- 
        node[near end,right=0.2] {$2\vec{x}$}  (0.4,0.6);
    \draw [->, thick] (0,0) -- 
        node[near end,left=0.2] {$-2\vec{x}$} (-0.4,-0.6);
  \end{tikzpicture}
  \caption{Scalar Multiplication}
  \end{figure}

  Scalar multiplication has the following properties
  \begin{enumerate}
  \item $\alpha(\beta \vec{x}) = (\alpha \beta) \vec{x}$.
  \item $\alpha(\vec{x}+\vec{y}) = \alpha\vec{x} + \alpha\vec{y}$.
  \item $(\alpha + \beta)\vec{x} = \alpha\vec{x} + \beta\vec{x}$.
  \item $1\vec{x} = \vec{x}$.
  \end{enumerate}
\end{fact}

\begin{fact}[Inner Product]
  Given two vectors $\vec{x}$ and $\vec{y}$ their inner product (also
  known as dot product or scalar product) is defined as
  \[\dotprod{\vec{x}}{\vec{y}} = |\vec{x}||\vec{y}| \cos \theta\]
  where $|\vec{x}|$ and $|\vec{y}|$ denote the length of the vectors
  and $\theta$ denotes the angle between them.

  \begin{figure}[h]
  \begin{tikzpicture}[scale=3]
    \draw [->] (0,0) -- (0,1);
    \draw [->] (0,0) -- (1,0);
    \draw [->,thick] (0,0) -- 
        node [right=0.4] {$\vec{x}$} (10:1);
    \draw [->,thick] (0,0) -- 
        node [above=0.1] {$\vec{y}$} (50:1);
    \draw [->] (10:0.2) node [above=0.06] {$\theta$} arc (10:50:0.2);
  \end{tikzpicture}
  \caption{Inner Product}
  \end{figure}

  The inner product has the following properties
  \begin{enumerate}
  \item $\dotprod{\vec{x}}{\vec{y}}=\dotprod{\vec{y}}{\vec{x}}$.
  \item $\dotprod{(\vec{x}+\vec{y})}{\vec{z}}
    = \dotprod{\vec{x}}{\vec{z}} + \dotprod{\vec{y}}{\vec{z}}$.
  \item $\dotprod{(\alpha\vec{x})}{\vec{y}} =
    \alpha(\dotprod{\vec{x}}{\vec{y}})$.
  \item $\dotprod{\vec{x}}{\vec{x}}=|\vec{x}|^2$.
  \end{enumerate}
\end{fact}

\begin{xca}
  Prove that the inner product of two vectors is positive if they form
  an acute angle with each other, negative if they form an obtuse
  angle with each other and zero if they form a right angle with each other.
\end{xca}

\begin{xca}
  Consider two two-dimensional vectors $\vec{x}=\begin{bmatrix} x_1 &
    x_2 \end{bmatrix}$ and $\vec{y} = \begin{bmatrix} y_1 &
    y_2 \end{bmatrix}$. Suppose we define their inner product to be
  \[\dotprod{\vec{x}}{\vec{y}} = x_1 x_2 + y_1 y_2.\]
  Show that this gives the same answer as our earlier definition
  \[\dotprod{\vec{x}}{\vec{y}} = |\vec{x}||\vec{y}|\cos \theta.\]
\end{xca}

\begin{xca}
  Let $\vec{x}$ be a nonzero vector and define the set
  \[S = \{\vec{y} \mid \vec{y} = \alpha\vec{x} \text{ for some $\alpha
    \in \Re$}\}.\]
  \begin{enumerate}
    \item What does the set $S$ look like in geometric terms?
    \item Let $\vec{z}$ be a given vector and let $\vec{p}$ be the
      point in $S$ closest to $\vec{z}$. Show that it must be the case
      that
      \[\dotprod{\vec{z}}{(\vec{z}-\vec{p})}=0.\]
      Interpret this condition geometrically.
    \end{enumerate}
\end{xca}
\section{Matrices}
  We assume that you know how to multiply two matrices, a matrix and a
  scalar a matrix and a vector.
\begin{fact}[Linearity of matrix-vector multiplication]
  Let $\mat{A}$ be a $n \times m$ matrix, $\vec{x}$ and
  $\vec{y}$ be $m$ dimensional vectors and let $\alpha$ be a scalar.
  Then the following is true,
  \begin{enumerate}
    \item $\mat{A}(\vec{x}+\vec{y})=\mat{A}\vec{x} + \mat{A}\vec{y}$.
    \item $\mat{A}(\alpha\vec{x}) = \alpha (\mat{A}\vec{x})$.
  \end{enumerate}
\end{fact}

For vectors the basic operations are addition and scalar
multiplication. The above Fact says that multiplication by a matrix
``preserves'' these operations in the sense that if we think of matrix
multiplication as a black box that takes in one vector and gives us
another then any relation between vectors that can be expressed in
terms of sums and scalar products is preserved when the vectors are
passed through this black box.

\begin{fact}[Invertibility]
  For a square matrix $\mat{A}$ the following conditions are
  equivalent
  \begin{enumerate}
  \item There is a matrix $\mat{B}$ such that 
    \[\mat{A}\mat{B}=\mat{B}\mat{A}=\mat{I}.\]
  \item $\mat{A}\vec{x}=0$ implies $\vec{x}=\vec{0}$.
  \item $\mat{A}\vec{x}=\mat{A}\vec{y}$ implies $\vec{x}=\vec{y}$.
  \item $\det(\mat{A}) \ne 0$.
  \end{enumerate}
\end{fact}

\chapter{Differentiation: single variable}
In this chapter we shall assume that all functions are defined over
all of $\Re$ and take values in $\Re$.

\begin{defn}
  Given a function $f$ and a point $x$ if there is a
  number $f'(x)$ such that
  \[f'(x) = \lim_{y \to x} \frac{f(y)-f(x)}{y-x}\]
  then we say that $f$ is differentiable at $x$ and its derivative
  at $x$ equals $f'(x)$.
\end{defn}

We can rearrange the above expression and say that the
function~$f$ is differentiable at $x$ if and only if there is a
number $f'(x)$ such that 
\begin{equation}\label{eq:diff1:linapprox}
\lim_{y \to x} \frac{f(y)-[f(x)+f'(x)(y-x)]}{y-x}
=0.
\end{equation}

Here, for given~$x$ the function
\[g(y) = f(x)+f'(x)(y-x)\] is a equation of a straight line
passing through $(x,f(x))$ with the slope $f'(x)$. The
numerator in~\eqref{eq:diff1:linapprox} is the error we make
if we try to approximate $f(y)$ by $g(y)$. The denominator
is the distance between $y$ and
$x$. 

So~\eqref{eq:diff1:linapprox} says that as $y$ comes
closer to $x$ not only does the error in approximating
$f(y)$ by $g(y)$ go to zero (you can check that this
condition is met by any line passing through $(x,f(x)$)
but also that the ratio of this error to the distance
between $y$ and $x$ also goes to zero (which is a harder
condition to meet.). Only if the second condition is also met
do we say that the function is differentiable. Thus a
function is differentiable if and only if it has a very good
linear approximation. As a result, when investigating the
behaviour of a function near a point, a differentiable
function can often be replaced by its tangent line.

\begin{thm}[Chain Rule]
  Given two functions $f$ and $g$ define a new function $h$ by
  \[h(x)=f(g(x)).\]
  Suppose $g$ is differentiable at some point $x_0$ and $f$ is
  differentiable at $g(x_0)$ then $h$ is differentiable at $x_0$ and 
  \[h'(x_0) = f'(g(x_0))\cdot g'(x_0).\]
\end{thm}

\begin{thm}[Taylor's theorem; first order]
  If $f$ is a twice-differentiable function then for any two points
  $a$ and $b$ there exists a point $\xi$ such that $a \le \xi \le b$
  and
  \[f(b) = f(a) + f'(a)(b-a) + \frac{f''(\xi)}{2}(b-a)^2.\]
\end{thm}

\begin{xca}
  Argue carefully that for small $h$, $\ln(1+h)$ is approximately
  equal to $h$. What do you understand by ``approximately equal to''?
\end{xca}

\chapter{Optimization: single variable}
\begin{thm}\label{thm:opt1:interior}
  Suppose the function $f\colon \Re \to \Re$ attains its
  maximum on the set $D$ at the point $x$. If $x$ lies in
  the interior of $D$ and $f$ is differentiable at $x$ then
  it must be the case that
  \[f'(x)=0.\]
\end{thm}
\begin{xca}
  Prove Theorem~\ref{thm:opt1:interior}. 

  Hint: Let $x$ be as in the theorem. First show that the
  fraction
  \[\frac{f(y)-f(x)}{y-x}\]
  must be less than or equal to zero when $y>x$ and greater
  than or equal to zero when $y<x$. Now proceed remembering
  how inequalities behave under limits.
\end{xca}

Theorem~\ref{thm:opt1:interior} would also have been true if
we had replaced the word ``maximum'' by the word
``minimum''.

Theoem~\ref{thm:opt1:interior} gives only a necessary
condition for a maximum. However, in many economic applications
this is often all that we need in order to find the maximum. Any point
in the domain of $f$ must either fail one of the
preconditions in the theorem or must satisfy its
conclusion. Therefore the maximum of $f$ must occur at a
point which is either
\begin{enumerate}
\item a boundary point of $D$,
\item a point where $f$ is not differentiable,
\item or, a point where $f'(\cdot)=0$.
\end{enumerate}
In many economic problems the set of points which satisfy at
least one of the criteria above is quite small. Since the maximum of the
function on $D$ must occur among these points it is enough
to evaluate the objective function $f$ at each of these
points and choose the one(s) which give the highest value.

\begin{xca}
Maximise
\[x + \ln y\]
subject to
\[p_x x + p_y y = M,\, x\ge 0,\,y\ge 0,\]
where $p_x$, $p_y$ and~$M$ are positive constants.

Be careful to consider the possibility of corner
solutions. Your answer should express the optimal values of
$x$ and $y$ as a function of $p_x$, $p_y$ and~$M$.
\end{xca}

For concave $f$ a partial converse of
Theorem~\ref{thm:opt1:interior} is available.

\begin{thm}
  If the function $f$ is concave on a convex domain $D$ and
  $f'(x)=0$ for some $x \in D$ then $f$ reaches its maximum
  over $D$ at $x$.
\end{thm}

\chapter{Differentiation: many variables}
\section{Definition}
We continue with the idea that the derivative of a function
must give a good linear approximation to a function near a
point. Now suppose we have a function that maps points in
$\Re^n$ to points in $\Re^m$. Then the linear approximation
to it must also map points from $\Re^n$ to $\Re^m$. 

One way to map points in $\Re^n$ to $\Re^m$ is to multiply
the input vector by a $m \times n$ matrix. Matrix
multiplication is linear in the sense that sums and scalar
multiples are preserved by the operation, i.e.\ if $A$ is a
matrix, $\xi$ and $\eta$ are vectors and $\alpha$ is a
scalar then
\[A(\xi+\eta)=A\xi+A\eta,\qquad
A(\alpha\xi)=\alpha(A\xi).\]
In fact, one learns in linear algebra that all linear
function from $\Re^n$ to $\Re^m$ are equal to multiplication
by some $m \times n$ matrix.

We want our linear approximation to $f$ at some point $x$ to
give the same value as $f$ for the input $x$. To ensure this
we add an appropriate constant to our approximating
function. So, as a linear approximation we propose,
\[g(y) = f(x) + A(y-x)\]
for some $m \times n$ matrix $A$ still to be determined.

Now once again we define differentiability as the existence
of a matrix $A$ which will make this approximation good.

\begin{defn}\label{defn:diff_n}
  Let $f\colon \Re^n \to \Re^m$ be a function defined over
  all of $\Re^n$. Let $x \in \Re^n$. We say that $f$ is
  differentiable at $x$ if there exists a matrix $A$ such
  that
  \begin{equation}\label{eq:diffn:linapprox}
    \lim_{y \to x}
    \frac{\lvert f(y)-[f(x)+A(y-x)]\rvert}
    {\lvert y-x \rvert}=0.
  \end{equation}
  In this case $A$ is said to be the derivative of $f$ at
  $x$, a fact denoted by
  \[Df(x)=f'(x)=A.\]
\end{defn}

You should compare this definition
to~\eqref{eq:diff1:linapprox}, which is the corresponding
definition in the single variable case. You will notice that
in both cases we are demanding that the ratio of the
approximation error to the change in inputs go to zero. The
only difference is that since in the present case both the
approximation error and the change in the input are vectors
we need to consider the length of each (hence the $\lvert
\cdot \rvert$ in the definition) in order to get a
numerical measure of how large these errors and changes are.

While we have defined the condition that a matrix $A$ has
to satisfy in order to be the derivative of a function at a
point, we still don't know how to find out if there is any
matrix at all which satisfies this condition and if so how
to calculate this matrix. In the next section we describe a
situation where our knowledge of calculus in one variable
will allow us to answer both these questions.

\section{Continuously differentiable functions}
\subsection{Partial derivatives}
In this subsection we consider function which have only one
output, that is functions $f$ from $\Re^n$ to $\Re$. We can
imagine varying only one input of the function at a time and
observing the change in the output. Let's say that we vary
only the $j$-th input, holding the others constant. This
defines for us a function of one variable,
\[\phi(x_j) = f(x_1,\ldots,x_j,\ldots, x_n)\]
where the values of $x_i, i\ne j$ are held constant.
Remember that for different choices of these values we will
get different functions $\phi$.

Now suppose that $\phi$, considered as a function of one
variable, is differentiable. Then its derivative is called
the $j$-th partial derivative of the function $f$. Recalling
the definition of the derivative of a function of one
variable we have
\[D_j f = \frac{\partial f}{\partial x_j}
=\lim_{h \to 0} 
\frac{f(x_1,\ldots, x_j+h,\ldots, x_n) 
-f(x_1,\ldots,x_j,\ldots,x_n)}
{h}.
\]

Note that the value of this limit depends not only on $x_j$
but also on the constant values $x_i, i\ne j$ at which we
evaluate it.

The $j$-th partial derivative of $f$ allows us to construct
a linear approximation to $f$ provided we are considering
changes only in its $j$-th input. Next we try to put
together these partial approximations for different $j$ to build a
linear approximation which will work for any change in the
inputs.

\subsection{The class $C^1$.}
To help us draw diagrams, let us look at a function $f$
whose domain is two-dimensional. Suppose we are trying to
get a linear approximation to this function near a point
$\xi = \begin{bmatrix} \xi_1 & \xi_2 \end{bmatrix}$.

Consider another point $\eta = \begin{bmatrix} \eta_1 &
  \eta_2 \end{bmatrix}$. In general both the coordinates of
$\eta$ will differ from those of $\xi$. 

Now suppose we introduce an intermediate point $\gamma
= \begin{bmatrix} \gamma_1 & \gamma_2\end{bmatrix}
= \begin{bmatrix} \eta_1 & \xi_2\end{bmatrix}$ (See Figure~\ref{fig:diff-break-change}). Then the
change from $\xi$ to $\gamma$ involves a change in only the
first coordinate and that from $\gamma$ to $\eta$ involves
a change only in the second coordinate. So we can approximate
each of these changes using the corresponding partial
derivative.
\begin{align}
  f(\gamma)-f(\xi) &\approx  (D_1 f)_{\xi}(\gamma_1-\xi_1)
   = (D_1 f)_{\xi}(\eta_1-\xi_1)\label{eq:diff-breakc-x}\\
  f(\eta)-f(\gamma) &\approx  (D_2 f)_{\gamma}(\eta_2-\gamma_2)
   = (D_2 f)_{\gamma}(\eta_2-\xi_2)\label{eq:diff-breakc-y}
\end{align}
Here $D_1$ and $D_2$ denote the partial derivatives with
respect to the two coordinates respectively while the
subscripts $\xi$ and $\gamma$ denote the points at which
these partial derivatives are evaluated.

\begin{figure}
  \begin{tikzpicture}
    \coordinate (xend) at (5,0);
    \coordinate (yend) at (0,5);
    \draw [->] (0,0) -- (yend);
    \draw [->] (0,0) -- (xend);
    \coordinate (xi) at (1,1);
    \node[above] at (xi) {$\xi$};
    \coordinate (eta) at (4,4);
    \node[above] at (eta) {$\eta$};
    \coordinate (gamma) at (4,1);
    \node[right] at (gamma) {$\gamma$};
    \draw[->,thick] (xi) -- (eta);
    \draw[->,thick] (xi) -- (gamma);
    \draw[->,thick] (gamma) -- (eta);
    \draw[dashed] (gamma) |- node[below] {$\eta_1=\gamma_1$} (xend);
    \draw[dashed] (xi) |- node[below] {$\xi_1$} (xend);
    \draw[dashed] (xi) -| node[left] {$\gamma_2=\xi_2$} (yend);
    \draw[dashed] (eta) -| node[left] {$\eta_2$} (yend);
  \end{tikzpicture}
  \caption{Decomposing an arbitrary change into changes along coordinate axes.}
  \label{fig:diff-break-change}
\end{figure}

Adding~\eqref{eq:diff-breakc-x} and~\eqref{eq:diff-breakc-y}
we have
\begin{align*}
  f(\eta)-f(\xi) &\approx (D_1 f)_\xi(\eta_1-\xi_1) 
  + (D_2 f)_\gamma (\eta_2-\xi_2)\\
  &=\begin{bmatrix} (D_1 f)_\xi & (D_2 f)_\gamma \end{bmatrix}
  \begin{bmatrix} \eta_1-\xi_1 \\ \eta_2-\xi_2 \end{bmatrix}
\end{align*}
This begins to look very much like a linear approximation.
The only remaining problem is that the matrix
$\begin{bmatrix} (D_1 f)_\xi & (D_2 f)_\gamma \end{bmatrix}$
is a function of $\gamma$ whereas the matrix~$A$ in the
definition of the derivative (Definition~\ref{defn:diff_n})
must be a constant for a particular point at which the
derivative is being evaluated (in our case $\xi$). 

This last hurdle can be crossed if $D_2 f$ is a continuous
function since in that case $(D_2 f)_\gamma$ will be
approximately equal to $(D_2 f)_\xi$ and we can replace the
former by the latter to get a matrix which is a constant
given $\xi$. The argument can be made precise and gives us,

\begin{thm}
  Consider a function $f\colon \Re^n \to \Re$. If all the
  partial derivatives of $f$ exist and are continuous then
  $Df$ exists and
  \[Df = \begin{bmatrix} 
    D_1 f & D_2 f & \hdots & D_n f
    \end{bmatrix}.\]
\end{thm}
The result would not have been true without the assumption
of continuity on the partial derivatives.

We now want to generalise this result to functions with
multiple outputs, i.e.\ functions $f \colon \Re^n \to
\Re^m$. The way to do this is to think of each of the outputs
as one particular function from $\Re^n$ to $\Re$, apply our
previous result to each of these functions, and stack the
answers together. This gives us,
\begin{thm}\label{thm:c1-jacobian}
  Consider a function $f\colon \Re^n \to \Re^m$. Define
  functions $f_i, i=1,\ldots,m$ such that
  \[f(x) = \begin{bmatrix}
    f_1(x)\\
    \vdots\\
    f_m(x)
    \end{bmatrix}
  .\]
  Suppose further that for each of the functions $f_i$ all
  partial derivatives exist and are continuous. Then $Df$
  exists and is given by
  \[Df = \begin{bmatrix} 
    D_1 f_1 & D_2 f_1 & \hdots & D_n f_1\\
    D_1 f_2 & D_2 f_2 & \hdots & D_n f_2\\
    \hdotsfor{4}\\
    D_1 f_m & D_2 f_m & \hdots & D_n f_m
    \end{bmatrix}
    .\]
\end{thm}

Functions which satisfy the conditions of
Theorem~\ref{thm:c1-jacobian} are called continuously
differentiable functions or said to belong to the class
$C^1$. For such $C^1$ functions
Theorem~\ref{thm:c1-jacobian} not only tells us that the
derivative exists but also tells us how to calculate it. 

In fact it is further possible to show that for $C^1$
functions $Df(x)$ is a continuous function of $x$, which is
how this class gets its name. 

\begin{xmpl}
  Consider the function $f\colon \Re^2 \to \Re^3$ given by
  \[f(x) = \begin{bmatrix}
    x_1+x_2\\
    x_1 - x_2\\
    x_1 x_2\\
   \end{bmatrix}.\]
   So for example $f(\begin{bmatrix} 2 & 3 \end{bmatrix})
   = \begin{bmatrix} 5 & -1 & 6 \end{bmatrix}$.

   The component functions are
   \begin{align*}
     f_1(x) &= x_1+x_2\\
     f_2(x) &= x_1-x_2\\
     f_3(x) &= x_1 x_2
   \end{align*}

   Each of these component function has both partial
   derivatives. They are,
   \begin{align*}
     D_1f_1 = 1,\qquad & D_2 f_1 = 1\\
     D_1f_2 = 1,\qquad & D_2 f_2 = -1\\
     D_1f_3 = x_2,\qquad & D_2 f_3 = x_1
   \end{align*}
   
   We see that each of these partial derivatives are
   continuous. Therefore, according to
   Theorem~\ref{thm:c1-jacobian}, $Df$ exists and is given by
   \[Df=
   \begin{bmatrix}
     1 & 1\\
     1 & -1\\
     x_2 &  x_1
   \end{bmatrix}
   .\]
\end{xmpl}

\section{The Chain Rule}
\begin{thm}\label{thm:chainrule-n}
  Let $h\colon \Re^n \to \Re^m$ and $g\colon \Re^m \to
  \Re^q$ be differentiable functions. Suppose we define
  $f\colon \Re^n \to \Re^q$ by
  \[f(x)=g(h(x)).\]
  Then $f$ is also differentiable and its derivative is
  given by
  \begin{equation}
    \label{eq:chainrule-n}
    Df(x) = [Dg(h(x))][Dh(x)].
  \end{equation}
\end{thm}

In~\eqref{eq:chainrule-n}, $[Dh(x)]$ is a $m \times n$
matrix and $Dg(h(x))$ is a $q \times m$ matrix and $Df$ is
their product.

\section{Directional derivatives and gradients}
Given a function $f\colon \Re^n \to \Re$ and a point $x^*$,
how does the value of $f$ change if we move a
little bit in the same direction as some vector $v$? If $f$ is
differentiable we can use calculus to investigate this
question.

First we define the function 
\[\phi(\alpha)=x^* + \alpha v\]
Then $\phi(0)=x^*$ and as $\alpha$ increase from $0$ we move
away from $x^*$ in the direction given by $v$. 

By writing the definition of $\phi$ in terms of components
and taking derivatives we see that $D\phi(\alpha) =
v$ for all $\alpha$.

To see how this movement along $v$ affects $f$ we define another
function
\[\psi(\alpha) = f(\phi(\alpha))\]

The derivative of $\psi$ gives us the rate of change of $f$
as we move in the direction $v$. It is called the
directional derivative of $f$ in the direction $v$ and
denoted $D_vf$. 

If $f$ is differentiable
then the Chain Rule gives us,
\begin{equation}\label{eq:dir-deriv-chain}
D_vf(x^*)=\psi'(0) = Df(\phi(0))D\phi(\alpha)=Df(x^*)v
\end{equation}

From the definition of the derivative
\begin{equation}\label{eq:lin-approx-dir}
f(x^*+\alpha v) - f(x^*) = \psi(\alpha) - \psi(0) \approx
\psi'(0)\alpha = \alpha Df(x^*)v
\end{equation}

Thus, whether moving a little bit in the direction $v$
increases or decreases the value of $f$ depends on whether
$Df(x^*)h$ is positive or negative in sign. There is an
alternative way of writing this which helps in understanding
the geometry of the situation.

Since $f$ maps $\Re^n$ to $\Re$, $Df$ is the $1 \times n$
matrix $\begin{bmatrix} D_1f & \hdots & D_n f\end{bmatrix}$
and 
\[(Df)v = (D_1f) v_1+\cdots+(D_nf) v_n\]

We define the transpose of $(Df)$ as the gradient of $f$
(denoted $\nabla f$). That is,
\[\nabla f = \begin{bmatrix}
D_1 f\\
\vdots\\
D_n f
\end{bmatrix}.\]

Now we can write $(Df)v$ as the scalar product $\dotprod{(\nabla
f)}{v}$. The advantage of doing this is that we can use our
geometrical intuition about the scalar product.
Equation~\eqref{eq:lin-approx-dir} becomes,

\[
f(x^*+\alpha v) - f(x^*)  \approx
\psi'(0)\alpha = \alpha Df(x^*)v = \alpha[\dotprod{(\nabla
  f)}{v}].\]

So a movement along $v$ approximately increases, leaves
unchanged or decreases $f$ depending on whether $v$ makes an
acute, right or obtuse angle with $\nabla f$.

\section*{Exercises}
\begin{xca}
  Consider the function $f\colon \Re^3 \to \Re^2$ given by
  \[f(x) = \begin{bmatrix}
    x_1x_2\\
    x_2x_3\\
    \end{bmatrix}.
   \]
   Does $Df$ exist? If so, calculate its value.
\end{xca}
\begin{xca}
  Consider the functions $f\colon \Re \to \Re^2$ and
  $g\colon \Re^2 \to \Re$ given by,
  \[f(t)=\begin{bmatrix}
    t\sin(t)\\
    t\cos(t)
    \end{bmatrix},\qquad
    g(x) = x_1^2 + x_2^2
  \]
  \begin{enumerate}
  \item Calculate $Df$ and $Dg$.
  \item Let $h\colon \Re \to \Re$ be given by $h(t) =
    g(f(t))$. Give an explicit formula for $h(t)$.
  \item Calculate $Dh$ using
    \begin{enumerate}
    \item the explicit formula derived above.
    \item the Chain Rule.
    \end{enumerate}
  \end{enumerate}
\end{xca}

\begin{xca}
  If $f$ and $g$ are continuously differentiable functions
  from $\Re^n$ to $\Re$.
  \begin{enumerate}
    \item\label{lab:diffn:fg1} Show that
      \[D(fg) = f(Dg) + g(Df).\]
    \item Verify the result for
      \[f(x) = x_1+x_2,\qquad g(x) = x_1 - x_2.\]
    \item (Optional, harder) Would the result in
      (\ref{lab:diffn:fg1}) still hold if we assumed only that
      the functions were differentiable, rather than being
      continuously differentiable?
    \end{enumerate}
\end{xca}

\section*{References}
Simon \& Blume, Chapter 14.

\chapter{Optimization: many variables}
\section{Unconstrained optimization}
\begin{thm}\label{thm-nvar-ucopti}
  If $f\colon \Re^n \to \Re$ attains its maximum  on the
  set~$D$ at a point $x^*$, if $x^*$ is an interior point of $D$
  and $f$ is differentiable at $x^*$ then it must be the case
  that
  \[Df(x^*)=0\]
\end{thm}
\begin{proof}
  Let $h$ be an arbitrary nonzero vector. Define the
  functions
  \begin{align*}
    \psi(\alpha) & = x^*+\alpha h\\
    \phi(\alpha) &= f(\psi(\alpha))
  \end{align*}
  We restrict $\alpha$ to a small enough interval around
  zero such that $\psi(\alpha) \in D$. This is possible since
  $x^*$ is in the interior of $D$. Since $x^*$ is the point
  where $f$ attains its maximum in $D$, it must be the case
  that $\phi$ attains its maximum at $0$ which is an
  interior point of its domain. Therefore by
  Theorem~\ref{thm:opt1:interior} it must be be case that
  \[\phi'(0) = 0\]
  

  By the chain rule $\phi'(0) = Df(x^*)D\psi(0)$. But a
  direct calculation shows us that $D\psi = h$. Hence we
  have
  \[[Df(x^*)]h=0.\]
  The above is true for any $h$. This is possible only if
  \[Df(x^*)=0.\]
\end{proof}

Translated into the language of gradients the above theorem
claims that
\[\nabla f(x^*)=0.\]
Suppose on the contrary that $\nabla f(x^*) \ne 0$. Consider the
point
\[y = x^* + \alpha\nabla f(x^*)\]
Since $x^*$ is in the interior of $D$, we can ensure $y \in
D$ by choosing $\alpha$ small enough. Then from the
definition of the derivative we have
\begin{align*}
f(y) &\approx
f(x^*) + \dotprod{\nabla f(x^*)}{(y-x^*)}\\
&=f(x^*) + \alpha (\dotprod{\nabla f(x^*)}{\nabla f(x^*)})\\
&\text{[Since $y = x^*+\alpha\nabla f(x^*)$]}\\
&=f(x^*) + \alpha \lvert \nabla f(x^*) \rvert^2\\
&>f(x^*)
\end{align*}
Since $y$ gives a higher value of the function than $x^*$,
$x^*$ could not be the maximum point. 

What stops this argument from being a proof is the
approximately equals sign. Without accounting for the the
approximation error we cannot be sure that
$f(y)>f(x^*)$. Our actual proof bypasses this difficulty.

\begin{xca}
  Let $A$ be a matrix such that $Ax=0$ for all vectors
  $x$. Show that it must be the case that $A=0$.
\end{xca}

\section{Equality constraints}
\begin{thm}\label{thm:equality:foc}
Let $f\colon\Re^n \to \Re$ and $g\colon \Re^n \to \Re$ be
continuously differentiable functions. Consider the problem
of maximising $f(x)$ subject to the constraint $g(x)=c$ for
some $c$.

If the maximum among the points satisfying the constraint is
attained at the point $x^*$ and $\nabla g(x^*) \ne 0$ then
there must exist a $\lambda \in \Re$ such that
\[\nabla f(x^*) =\lambda \nabla g(x^*).\]
\end{thm}

\begin{figure}
  \begin{tikzpicture}
     \draw [->] (0,0) node[below] {$x^*$} -- +(30:2) node[above] {$\nabla g$} ;
     \draw [->] (0,0) -- +(60:2) node[above] {$\nabla f$} ;
     \draw [->] (0,0) -- +(120:1) node[above] {$h$} ;


     \draw [->] (5,1) node[left] {$x^*$} -- +(30:2) node [above] {$\nabla g$} ;
     \draw [->] (5,1) -- +(-30:2) node [above] {$\nabla f$} ;
     \draw [->] (5,1) -- +(-60:1) node [below] {$h$} ;

     \draw [->] (0,5) node[left] {$x^*$} -- +(30:2) node [above] {$\nabla g$} ;
     \draw [->] (0,5) -- +(270:2) node [below] {$\nabla f$} ;
     \draw [->] (0,5) -- +(-60:0.5) node [below] {$h$} ;

     \draw [->] (6,5) node[below] {$x^*$} -- +(30:2) node [above] {$\nabla g$} ;
     \draw [->] (6,5) -- +(180:2) node [above] {$\nabla f$} ;
     \draw [->] (6,5) -- +(120:0.5) node [above] {$h$} ;
  \end{tikzpicture}
  \caption{Points which cannot be a maximum (Theorem~\ref{thm:equality:foc})}
  \label{fig:equality:foc}
\end{figure}

Theorem~\ref{thm:equality:foc} says that at an optimal point
the gradient of the objective function and that of the
constraint function must be collinear. If they are not, then
as Figure~\ref{fig:equality:foc} illustrates, there will be
a direction $h$ that will make a right angle with $\nabla g$ and
an acute angle with $\nabla f$. In that case, assuming the
magnitude of $h$ to be small enough we will have
\begin{align*}
  g(x^*+h)- a &= g(x^*+h)-g(x^*) \approx \dotprod{\nabla
    g(x^*)}{h} = 0\\
  f(x^*+h)-f(x^*) &\approx \dotprod{\nabla f(x^*)}{h} > 0
\end{align*}
[We have used that fact that $g(x^*)=a$ since $x^*$
satisfies the constraint.]

If we could have replaced the approximate equalities in the
above relations by exact equality we would have produced a
proof that $x^*$ could not be a constrained maximizer since
$x^*+h$ also satisfies the constraint and gives a higher
value of the objective function.

However, since we cannot make such a replacement of
approximate equalities by exact equality, an actual proof of
Theorem~\ref{thm:equality:foc} must be more complex. You can
find such proofs in the references.

The condition $\nabla g \ne 0$ is a technical requirement
that is usually satisfied in economic applications.

The scalar $\lambda$ in the theorem is called a Lagrange
multiplier.

\section*{Exercises}
\begin{xca}
  Maximise $x+y$ subject to $x^2+y^2=1$.
\end{xca}

\begin{xca}
  Maximise $(x-4)^2-(y-4)^2$ subject to $x^2=0$.
\end{xca}

\begin{xca}
  Argue that minimising $f(x)$ subject to $g(x)=c$ is
  equivalent to maximising $-f(x)$ subject to $g(x)=c$.
\end{xca}
\end{document}